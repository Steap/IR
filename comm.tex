\section{Communication costs}
Sending messages between the nodes of a cluster can have a significant cost.
Autotuned applications can therefore take advantage of determining these
communication costs : basically, mapping processes which need to exchange data
very often to processors for which the communication cost is low should improve
the performance of the application\cite{MPIPP}.

Servet presents a 3-step algorithm that should be able to provide relevant data
about communication costs to applications.

\subsection{Communication layers}
First, Servet defines the concept of "communication layers". A communication
layer is a set of pairs of cores for which the communication costs are similar.
They start by sending a message between cores of each pair of cores in the
cluster, then group similar pairs into a common layer, using as many layers as
required.

The size of this message is equal to the L1 cache size. Does that always make
sense ? Sending messages larger than the L1 cache but still smaller than the
L2/L3 cache might be interesting to group processors which share L2/L3 cache.
Plus, sending messages larger than the L1 cache (which is usually very small)
would be closer to what will really happen when running multi-core applications.
Nothing is said about this in the article, and maybe the best idea is indeed to
use the L1 cache size. Still, it would be interesting to test other sizes and
compare the results.

\subsection{Point-to-point communication}
For each layer, the communication performance is estimated by sending messages
(with different sizes) 
% Citer 9

\subsection{Scalability}
In all these tests, only one message is sent at a time. In real application, it
is very likely that many messages will be sent at once, which means that the
current benchmark still has to make sure the system is highly scalable. It
consists in comparing the latency of a single message and the latency when all
the cores in a layer send one message.

If it is much more expensive to send N messages rather than one, then it will be
possible for autotuned applications to gather messages.
